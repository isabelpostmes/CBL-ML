\section{Methodology}

\label{sec:methodology}

Explain machine learning methodology

\subsection{General approach}

The main ingredients of our road to success will be:
\begin{enumerate}
    \item Data preparation, to set up the raw data for training.
    \item Neural network formulation, where the network architecture, optimization algorithm and loss function are defined.
    \item Minimization, which allows an efficient optimization on the three-dimensional input.
    \item Validation, where the optimal distribution of weights is defined by a stopping criterion and the goodness of the fit is quantified. 
    \item Interpolation and extrapolation, which is the unique selling point of this procedure. 
\end{enumerate}

\subsection{Data preparation}
The four data sets contain spectra that are recorded over a different energy range. As we are only interested in the ZLP region, where the intensity is different from zero, we drop all data that is outside $[-0.1, 0.5]$ eV. This way we capture all the relevant data, discard a big region with zero intensity and we have data for each of the sets over the entire energy range. \\
We then evolve our data from the original scale to a scale at which we can ease the optimization. Unscaled input variables can result in a slow or unstable learning process, whereas unscaled target variables on regression problems can result in exploding gradients causing the learning process to fail \cite{juan}. The energy loss is scaled between [$-1,1$] and all input spectra are scaled between [$0.1, 0.9$], where 0.9 corresponds to the maximum value of the intensity for each separate data set ($I_{max}$, table \ref{table:vacuum}). 

\subsubsection*{Observables}
Central values and errors are calculated for each of the data sets separately, as we expect different outcomes for different exposure time and beam energy. The discretization technique used to assign mean and standard deviation to the data points is Equal Width Discretization (EWD) \cite{Dash:2011}, a simple discretization method that divides the range of observed values for a feature into k equal sized bins. The intervals are computed by 
$\Delta E = (E_{max} - E_{min}) / k$. The value of k is chosen empirically such that the number of bins is high enough to not lose valuable information, but still contains a sufficient number of data points to calculate the uncertainties.\
 Within each energy bin $\Delta E$, the median and variance of all data points within this bin are determined and returned to the original data points. This way, each data point is a vector $[dE, D_i, \sigma_i]$ where dE is the original energy loss; $D_i$ and $\sigma_i$ are the median and std of the bin where this point belongs to. 

\subsubsection*{Monte Carlo sampling}
Error propagation from experimental data to the fit is performed by Monte Carlo sampling of the training data. In a Monte Carlo approach, any statistical property of the function with error - in our case the ZLP - can be derived from a Monte Carlo sample of replicas of the function. By the generation of $N_{rep}$ replicas of the individual training points, a multi-Gaussian distribution is obtained centered at $D_i$ with a standard deviation equal to the corresponding error $\sigma_i$. \\
To be precise, given a training data point
\begin{equation}
    Y = Y(dE, D_i,\sigma_i), 
\end{equation} a set of $k= 1,2,..,N_{rep}$ artificial training points is generated by adding a stochastic noise signal on top of the data: 
\begin{equation}
    Y^{(k)} = Y(dE, D_i,\sigma_i) + r^{(k)}\sigma_i.
\end{equation}
The variables $r^{(k)}$ are normally distributed random numbers, such that each element in the Monte Carlo set is a fluctuation around the central value of the experimental data: each replica $k$ contains as many data points as the original set. \\

The value of $N_{rep}$ should be chosen in such a way that the set of replicas models the probability distribution of original training data faithfully. A comparison of the central values and errors of the artificial set with the original data is shown in Fig \ref{mcvacuum} for samples of 100, 500, 1000 and 5000 replicas. The results from a more quantitative comparison can be observed in table \ref{tablemcvacuum}.

\begin{figure}[H]
    \centering 
    \includegraphics[width=160mm]{plots/Montecarlovacuum.png}
    \caption{Scatter plot of training data vs. Monte Carlo replicas central values (left) and errors (right). }
    \label{mcvacuum}
\end{figure}
\begin{table}[H]
\centering
\begin{tabular}{|l|ll|ll|}
\hline
Nrep & $\textless{\mu}\textgreater{}$ {[}eV{]} & r{[}$\mu${]} & $\textless{\sigma}\textgreater{}$ {[}eV{]} & r{[}$\sigma${]} \\ \hline
100  & 0.10259                           & 0.999919   & 0.63618                                  & 0.996990      \\ \hline
500  & 0.10258                              & 0.999982   & 0.63991                                  & 0.999388      \\ \hline
1000 & 0.10258                              & 0.999991   & 0.64013                                  & 0.999590      \\ \hline
5000 & 0.10258                              & 0.999997   & 0.63980                                  & 0.999776      \\ \hline
\end{tabular}
\caption{Comparison between the artificial and the original central values and errors. $\textless{\mu}\textgreater{}$ and $\textless{\sigma}\textgreater{}$ are the expectation value of the median and errors respectively. The scatter correlation $r$ gives the spread around the straight line. $\textless{\mu^{exp}}\textgreater{}$ = 0.10258, $\textless{\sigma^{exp}}\textgreater{}$ = 0.63970. }
\label{tablemcvacuum}
\end{table}


\subsection{Neural network and fitting strategy}
The neural network architecture that we use is 3-10-15-5-1, enabling deep learning with three hidden layers. This choice is made empirically and it is motivated by the flexibility of the network to overlearn, which is necessary for the post-selection criterion, yet to keep the computation time reasonable. The use of a redundant architecture excludes a priori the bias that an underlearning model would bring forth. 

A schematic of the neural network graph can be observed in Fig. \ref{architecture}. 
\begin{figure}[H]
    \centering
    \includegraphics[width=90mm]{plots/architecture.jpg}
    \caption{Schematic representation of the neural network used for the prediction of zero loss peaks in vacuum. The input spectrum is a three-dimensional vector containing the energy loss, exposure time and beam energy. The input is passed through three successive fully connected layers before a one-dimensional output of the intensity is produced.}
    \label{architecture}
\end{figure}

$N_{rep}$ neural networks are trained on the Monte Carlo data, each on a different set of replicas. The networks are trained to minimize the uncorrelated $\chi^2$ per data point:

\begin{equation} \label{eq:chi2}
    E^{(k)} = \frac{1}{N_{dat}}\sum_{i=1}^{N_{dat}}\left(\frac{D_i^{MC(k)} - D_i^{pred(k)}}{\sigma_i}\right)^2, 
\end{equation}

where $D_i^{MC(k)}, D_i^{pred(k)}$ are the artificial Monte Carlo intensity and the predicted intensity in the $k$-th replica, $\sigma_i$ is the standard error associated to the data point. \\
The chi-squared method is the cornerstone of almost all fitting, as it is an intuitively reasonable measure of how well the predictions fit the data. If the model predictions are all within one standard deviation from the data, then the $\chi^2$ per data point takes a value roughly equal to 1. In general, if $\chi^2/N_{dat}$ is of the order 1, we can say that the fit is a good approximation to the real data. \\

We are not looking for the absolute minimum of the error function, but we rather search for an optimal stopping point to avoid overfitting. At this optimal stopping point, it reproduces the general features of the inputs, but not some accidental fluctuations. This point can be determined by the formulation of a stopping criterion, making it possible to, once the network completed the training, choose the parametrization of the network weights right before it entered the overlearning regime. The look-back stopping method has been widely used in the context of neural networks. \\
In this technique the training data is first shuffled and then divided into two subsets, usually split 80-20. The first and biggest set is the training set, used for updating the network weights and biases under the minimization of the error function (\ref{eq:chi2}). After each training epoch, the validation set is given as an input to the neural network and the error is monitored. During the initial phase of training, the error on both the training and validation set decrease, but at a certain point the network enters the overlearning regime and the validation error slowly starts to increase, while the training error further decreases. It is at this point that weights and biases are collected to make the prediction. This can only be determined in hindsight, hence the name 'look-back stopping'. One thing to keep in mind is to avoid early stopping in a local minimum rather than the global minimum of the validation error, which can be assured by setting the training length sufficiently long to reach the global minimum.

\subsection{Minimization}
The minimum of the fitting procedure is found using the backpropagation method. 

\subsection{Validation}

\subsubsection*{Closure Testing}
In order to validate if the computed error associated to each training point is reasonable, a method called closure testing provides insight \cite{Ball:2015oha}. A closure test ensures that uncertainties introduced by methodology are small compared to the generic experimental errors of the data. A basic requirement for a successful methodology is the ability of generalization, which means that the model works on widely different datasets without the need for manual fine-tuning.\\
The basic idea of closure testing is the following \cite{Ball:2015oha}: we take as general assumed form of the ZLP the smooth fit to the data and we create a set of 'perfect' pseudo data by sampling this smooth function at our data points. Smoothing of the data is realized by application of a Hanning window \cite{Essenwanger:1986}, a popular method in signal processing, based on the convolution of a scaled window with the signal. \\
Next, on each 'perfect' data point we impose a gaussian random number with size equal to the corresponding error from the original dataset. If we then train our neural network on these points, given that the pseudo-data are fluctuated on average by one $\sigma$ away from the 'perfect' value, by construction we should find the error function $\chi^2$ of the best fit will be approximately one. Results of the $\chi^2$ values with the closure test can be observed in Fig. \ref{closure1}. These values were obtained with the look-back method discussed in section \textit{ (..nowhere yet..),} which ensures that the final $\chi^2$ is taken at the optimum stopping point, before overlearning sets in.

\begin{figure}[H]
    \centering
    \includegraphics[width=70mm]{plots/chi2.png}
    \caption{Histogram of $\chi^2$ values obtained after training the neural network on 500 sets of artificial data created for closure testing.}
    \label{closure1}
\end{figure}

\subsection{Interpolation and extrapolation}

ZLP in sample, where i) we compare with the vacuum results, ii) we modify the training region to make sure we don't fit the signal from sample, 3) we perform the substraction and look for interesting features in the low-loss spectra

\section{Methodology}

The energy loss of the sample data covers a $[-0.3, 8]$ eV range. Since the ZLP feature is several orders of magnitude bigger than the sample features, we take the logarithm of the intensity to train the network. 
 
\subsection{Data processing}
\begin{figure}[H]
    \centering
    \includegraphics[width=70mm]{plots/closure.png}
    \caption{Histogram of $\chi^2$ values obtained after training the neural network on 500 sets of artificial data created for closure testing.}
    \label{closure2}
\end{figure}

\subsection{Training regions} \label{sec:regions}
It is important to make an informed choice as to the energy range of the input data over which the neural network is trained. Figure \ref{ranges} below is a schematic low-loss spectrum showing five relevant energy ranges \cite{Reed:2002}. Region (a) doesn't contain loss electrons and contributions solely come from dark counts in the detector. Range (b) and (c) together form the 'ZLP-region', with the left- and right hand side of the ZLP respectively. Since we make use of a monochromated electron beam, the energy distribution is approximately symmetric around zero and range (c) will be the mirrored version of range (b) until a certain energy limit, which we will call $dE_1$ (we will elaborate on this assumption in section \ref{sec:subtraction}. Range (e) sets in at higher energy loss $dE_2$ and represents the almost pure loss signal, with essentially negligible ZLP contributions. Region (d) is the transition in which the ZLP tail gradually gives way to the loss signal. It is this region that is specifically of interest, as it contains the low-loss features of the sample.  Our objective is to automatically determine the right values for $dE_1$ and $dE_2$ before training the neural network.

\begin{figure}[H]
    \centering
    \includegraphics[width=100mm]{plots/ranges.png}
    \caption{Schematic illustration of the relevant energy ranges used in the model}
    \label{ranges}
\end{figure}

Before training, a window is to be applied to the training inputs to an upper boundary $dE_1$, to ensure the network trains on data of the ZLP solely. In order to attain a model that meets $log(I_{ZLP}) \rightarrow 0$ as $dE \rightarrow \infty$, pseudo data is to be added for $E>dE_2$ where the ZLP contribution becomes negligible. The model is trained on data for $E<dE_1$ and $E>dE_2$, the interpolation region $dE_1 < E < dE_2$ contains the predictions of interest. \\

\subsection{Derivatives for preliminary feature identification}

Using first and second derivatives is a feature extraction method to achieve a relatively high accuracy with a low computational complexity. The slope of any function is described by the first derivative, therefore the second derivative is essentially the change in slope, representing the curvature of the signal. This can be useful to determine the boundary from the ZLP to the sample region. \\

In an ideal microscope the electron beam would be perfectly monochromatic, correspondingly the ZLP would appear as a delta function in an EEL spectrum \cite{Rafferty:2000}. In practice the ZLP has a finite width defining the energy resolution of the system. At some energy loss the contribution to the sample kicks in and it is at this point that the intensity profile, earlier monotonically decreasing, will have its first local minimum and correspondingly the first derivative will cross zero. It is this point, previously defined as $dE_2$, that marks the transition from the interpolation to the sample region. The first crossing of $\frac{d}{dE}log(I)$ with zero differs slightly between the individual spectra: we take as $dE_2$ the lowest of these, as we can say with certainty that the intensity of the sample has kicked in at the first notice of a local minimum. For this set of data, we find $dE_2$ = $1.468$ eV (Fig. \ref{bound}). 

\begin{figure}[H]
    \centering 
    \includegraphics[width=170mm]{plots/derivatives.png}
    \caption{First and second derivatives and log derivatives of the nine experimental spectra. All functions are normalized to the maximum spectrum intensity to make sense of the relative change. The first crossing of the log derivative with zero is \textbf{1.468} eV and marks the transition to the sample region ($dE_2$). Before taking each derivative, smoothing by means of a Hann window \cite{hann} was applied to remove noise and reveal underlying trends. }
    \label{bound}
\end{figure}

From changes in the normalized second derivatives in Fig \ref{bound}, one is able to define significant features appearing in the spectrum. 

By stating that the ideal zero loss peak is symmetric around zero up to energy loss $dE_1$, hypothetically we could determine the value of $dE_2$ by mirroring the left-hand tail to the right side and see when the two significantly start to deviate. In practice however, the ZLP is not symmetric due to various external factors such as astigmatism \cite{astigma} and virtual scattering events \cite{rafferty}, which makes direct reflection of the left-hand tail unfavorable. \\
The value of $dE_1$ will be chosen as $dE_2 / a$, with $a$ a positive integer. The choice of $a$ is empirical and will be cross-validated. \\ 

After the choice on $dE_1$ and $dE_2$, the set of training data (TD) will be prepared taking the following steps:
\begin{enumerate}
    \item Keep only sample data (SD) inside the window [$dE_{min}, dE_1$]
    \item Create pseudo data (PD) with $log(I)$=0 in range [$dE_2, dE_{max}$]
    \item TD = SD + PD
\end{enumerate}

\subsection{Monte Carlo sampling}
The value of $N_{rep}$ should be chosen in such a way that the set of replicas models the probability distribution of original training data faithfully. A comparison of the central values and errors of the artificial set with the original data is shown in Fig \ref{mc} for samples of 100, 500, 1000 and 5000 replicas. The results from a more quantitative comparison can be observed in table \ref{tablemc}.

\begin{figure}[H]
    \centering 
    \includegraphics[width=160mm]{plots/MC.png}
    \caption{Scatter plot of training data vs. Monte Carlo replicas central values (left) and errors (right). }
    \label{mc}
\end{figure}

\begin{table}[H]
\centering
\begin{tabular}{|l|ll|ll|}
\hline
Nrep & \textless{}$\mu$\textgreater {[}eV{]} & r{[}$\mu${]} & \textless{}$\sigma$\textgreater {[}eV{]} & r{[}$\sigma${]} \\ \hline
100  & 11.35525                              & 0.99998925   & 0.17783                                  & 0.96617604      \\ \hline
500  & 11.35693                              & 0.99999808   & 0.17799                                  & 0.99172921      \\ \hline
1000 & 11.35578                              & 0.99999894   & 0.17812                                  & 0.99527003      \\ \hline
5000 & 11.35643                              & 0.99999979   & 0.17807                                  & 0.99920997      \\ \hline
\end{tabular}
\caption{Comparison between the artificial and the original central values and errors. $\textless{\mu}\textgreater{}$ and $\textless{\sigma}\textgreater{}$ are the expectation value of the median and errors respectively. The scatter correlation $r$ gives the spread around the straight line. $\textless{\mu^{exp}}\textgreater{}$ = 11.35623, $\textless{\sigma^{exp}}\textgreater{}$ = 0.17805. }
\label{tablemc}
\end{table}


\subsection{Parametrisation of in-vacuum spectra}


The first part of this studies is dedicated to the prediction of the shape of the zero loss peak when recorded in vacuum. We construct a generalized N-dimensional model which predicts the ZLP from N arguments. On this basis, we can successfully predict the shape of the ZLP with inter- and extrapolation with respect to each of the input variables. 

The predicted intensity at any point of the spectrum depends predominantly on the energy loss. We expect the highest intensity at zero loss and the number of counts rapidly decreasing as the energy loss increases. It is common to quantify this decrease in the Full Width Half Maximum (FWHM). Two variables that greatly account for the overall intensity of the ZLP are the exposure time and operating beam current, which both have a positive correlation on the number of counts. \\
Also other factors can account for changing ZLP intensities, such as aperture width and parameters of the electron gun and lenses. These variables could be given as inputs as well, however we expect that these are less relevant than the three input variables mentioned above. Therefore our default model will be built on these three arguments: energy loss, exposure time and beam current. We construct a 3-dimensional model with these continuous inputs and output the ZLP intensity. This model could be extended to an arbitrary dimension N, as long as information on the N input variables is available for training.



\subsection{Parametrisation of in-sample spectra}


\subsection{ZLP subtraction}
\label{sec:subtraction}

As mentioned before in section\ref{sec:regions}, the shape of a monochromated peak is approximately symmetric up to energy loss $dE_1$ (this should be checked with the vacuum recorded peak). We can determine the value of $dE_1$ by finding the energy loss where the mirrored ZLP starts to deviate from the full spectrum. It is this value that will be used as the upper boundary of the training window for the NN. \\

Repetitive training of the neural network on each set of MC pseudo data yields a prediction that is distributed with a mean and std corresponding to the mean and error of the original training set. For each replica, the predicted ZLP is subtracted from the individual original spectra to obtain one set of subtractions. Repeating this procedure yields a collection of $N_{rep}$ subtractions for each original spectrum, over which statistical properties such as median and variance can be calculated. \\

Direct subtraction of the predicted ZLP on each of the original spectra doesn't yield a physical result, as the intensity of the samples in range (a)-(c) (Fig. \ref{ranges}) doesn't coincide between themselves and the predicted ZLP. In order for the subtracted spectrum to make physical sense, we impose that the ZLP to be subtracted will be a smooth transition from the exact replicate of the ZLP in region (a)-(c) to the prediction from the neural network in region (d),(e). This way, we keep only the prediction of the region of our interest. The smooth matching of the ZLP will be the following:
\begin{align}
 dE &< dE_0          &   I_{ZLP}^{(k)} &= I_{orig}\\
 dE_0 &< dE < dE_1   &  I_{ZLP}^{(k)} &= I_{NN}^{(k)} \times  \left( 1 - \exp(-(dE - dE_1)^2/\delta^2)\right) \\
 dE_1 &< dE < 10 \cdot dE_2 &  I_{ZLP}^{(k)} &= I_{NN}^{(k)}\\
 dE &> 10 \cdot dE_2 &   I_{ZLP}^{(k)} &= 0
\end{align}
Here $dE_0$ is the start of the smooth transition from ZLP to NN region given by eq. $(5.4)$, with $dE_0<dE_1$  and $\delta \ll |dE_0 - dE_1|$. 

\begin{figure}[H]
    \centering
    \includegraphics[width=120mm]{plots/matching.png}
    \caption{Representation of the smooth transition from $I_{ZLP} = I_{orig}$ for $dE<dE_0$ to$I_{ZLP} = I_{NN}$ for $dE>dE_1$. }
    \label{fig:my_label}
\end{figure}



\subsection{Ratios for eventual feature identification}

In order to compare the level of desired signal from the sample to the level of the modeled ZLP tail, we define two working definitions, first of which the sample-to-peak ratio (SPR):

\begin{equation}
    SPR_{i}^{(k)} = \frac{I_{orig, i} - I_{ZLP}^{(k)}}{I_{ZLP}^{(k)}}
\end{equation}
where $i$ is the number of the original spectra, $k$ runs over the number of replicas ($k=1,..,N_{rep}$) and $I_{ZLP}$ is the predicted ZLP for that replica. By summing over all replicas, the median and error of $SPR_i$ give a measure for the contribution of the sample to the total intensity profile at any point over the spectrum. \\

To verify the significance of the SPR, we introduce the sample-to-error ratio (SER):
\begin{equation}
    SER_{i}^{(k)} = \frac{I_{orig, i} - I_{ZLP}^{(k)}}{\sigma_{ZLP}}
\end{equation}
where $\sigma_{ZLP}$ is the error of the predicted ZLP calculated over all replicas. 
If the SER is bigger than a certain treshold (e.g. 5), the signal from the sample significantly excesses the background error. \\
Using the definitions of SPR and SER for each original spectrum, it is convenient to define three regions: the 'ZLP region' where $SPR \rightarrow0$, corresponding to region (a)-(c) (Fig. \ref{ranges}) ending at $dE_1$, the 'interpolation region' (d) between $[dE_1, dE_2]$ and the 'sample region' (e) where $SPR\rightarrow\infty$ as $I_{ZLP}\rightarrow0$.
