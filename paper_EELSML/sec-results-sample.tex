\section{Results sample}
\label{sec:results_sample}

In this section we present the main results the sample part.


\paragraph{Sample spectra}
%
In this work we will study the local electronic properties
via a low-loss region analysis of EELS taken on
WS$_2$ nanostructures.
%
WS$_2$ is a highly promising material exhibiting a wide range of 
possible applications for electronic and optical devices.
%
When WS$_2$ is thinned down to a single monolayer, its 
indirect band gap switches to a direct band gap of around 2 eV~\ref{table:bgvalues}.


\begin{table}[h]
  \caption{Literature values for the bandgap of monolayer and bulk WS$_2$}
  \begin{indented}
\item[]\begin{tabular}{@{}lll}
\br
Entry {[}reference{]}                       & Type & Band gap energy (eV) \\
\mr
\multirow{2}{*}{1 \cite{Gusakova:2007}} & ML   & 2.03                 \\
                                            & Bulk & 1.32                 \\
2 \cite{Kam:1982}                  & ML   & 1.79                 \\
                                            & Bulk & 1.35                 \\
3 \cite{Jo:2014}                 & ML   & 2.14                 \\
                                            & Bulk & 1.40                 \\ 
\br                                         
\end{tabular}
\end{indented}
\label{table:vacuumdata}
\end{table}




%
A collection of electron loss spectra acquired at different positions 
at the specimen is used to construct the neural network training inputs. 
%
These sets of data were obtained directly from the work of (... ref to Sonia).
%
A specimen image of the positions can be observed in figure~\ref{fig:ws2positions}.  
This nanostructure exhibits flat layers with different thicknesses, which 
can be distinguished from the picture as color differences.
%
Energy loss spectra obtained at positions 1-3 are vacuum recordings, 
positions 4-13 represent in-sample data.




\subsection{Band-gap determination}


