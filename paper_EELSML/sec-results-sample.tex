\section{ZLP subtraction and bandgap determination in WS$_2$}
\label{sec:results_sample}

Following the discussion of the vacuum ZLP analysis, we now
present the application of our machine learning strategy to parametrise the ZLP
arising in EEL spectra recorded on specimens.
%
Specifically, we will analyse EELS measurements acquired on different regions
of the WS$_2$ nanoflowers reviewed in Sect.~\ref{sec:tmd}.
%
The resulting ZLP parametrisation will then be used to isolate the inelastic
contribution $I_{\rm inel}$ in each spectrum and determine the bandgap energy $E_{\rm BG}$ from
its behaviour in the onset region.

\subsection{Training dataset}
%
Low-magnification TEM images of two representative regions of
the WS$_2$ nanoflowers are displayed in Fig.~\ref{fig:ws2positions},
denoted as sample A and sample B.
%
In each image we indicate the specific locations where
EEL spectra have been recorded, including in-vacuum measurements taken
for calibration purposes.
%
Note that in sample B  the differences in contrast are related to the material
thickness, with higher contrast corresponding to thinner regions.
%
The two samples exhibit different structural morphologies: while sample A is composed
by a relatively thick region of WS$_2$, sample B corresponds to a thin petal region
with possibly overlapping terraces.
%
In other words, sample A is composed by bulk WS$_2$ while in sample B some specific regions
could be rather thinner, down to the few monolayers level.
%
This thickness information has been be determined
by means of the {\tt Digital~Micrograph}  software.
%
Further, the measurements on each sample have
been obtained with different electron microscopes and
operation settings, and for this reason
we analyse them independently.

%%%%%%%%%%%%%%%%%%%%%%%%%%%%%%%%%%%%%%%%%%%%%%%%%%%%%%%%%%%%%%%%%%%%%%%
\begin{figure}[t]
\begin{centering}
  \includegraphics[width=0.87\linewidth]{plots/Spectra_location.pdf}
  \caption{Low-magnification TEM images of two different regions of
    the WS$_2$ nanoflowers, denoted as sample A and sample B respectively.
    %
    In each image we indicate the locations where
    EEL spectra have been recorded, including the in-vacuum measurements taken
    for calibration purposes.
    %
    In sample B the difference in contrast is correlated to the material
    thickness, with higher contrast corresponding to thinner regions of the nanostructure.
    %
    The morphological differences between the two samples are discussed in the text.
  }
\label{fig:ws2positions}
\end{centering}
\end{figure}
%%%%%%%%%%%%%%%%%%%%%%%%%%%%%%%%%%%%%%%%%%%%%%%%%%%%%%%%%%%%%%%%%%%%%%%%%%

In Table~\ref{table:sampledata} we collect the most relevant properties of the spectra collected
in the locations indicated in Fig.~\ref{fig:ws2positions} using the same format as
in Table~\ref{table:vacuumdata}.
%
Note that since the of spectra from samples A and B
have been acquired with different microscopes, features of the ZLP
such as the FWHM are expected to be different.
%
From this table one can observe how  the ZLP for the spectra acquired on sample A exhibit
a FWHM about three times higher as compared to those of sample B
than the resolution obtained on the second set. 
%
This difference can be understood from the fact that the EELS spectra from sample B, unless those
from sample B,
were recorded with a TEM equipped with a monochromator.

%%%%%%%%%%%%%%%%%%%%%%%%%%%%%%%%%%%%%%%%%%%%%%%%%%%%%%%%%%%%%%%%%%%%%%%%%%%%%%%%%%%%%%%%%%%%%
%%%%%%%%%%%%%%%%%%%%%%%%%%%%%%%%%%%%%%%%%%%%%%%%%%%%%%%%%%%%%%%%%%%%%%%%%%%%%%%%%%%%%%%%%%%%%
\begin{table}[t]
  \begin{center}
            \renewcommand{\arraystretch}{1.50}
  \begin{tabular}{@{}ccccccccc}
\br
Set & $t_{\rm exp}$ {(}ms{)} & $E_{\rm b}$ {(}keV{)} & $N_{\rm sp}$ & $N_{\rm dat}$ & $\Delta E_{\rm min}$~(eV)  & $\Delta E_{\rm max}$~(eV)  & FWHM~(eV)  \\ 
\mr
A        &       ?       &        ?         &   6      &    1918    &     -4.1       & 45.5 & $ 0.47\pm0.01$  \\
B        &       ?       &    200 keV       &   10     &    2000    &     -0.9        & 9.1   & $ 0.16\pm0.01$ \\
\br
  \end{tabular}
    \end{center}
  \caption{\small Same as Table~\ref{table:vacuumdata} now for the EEL spectra taken on specimens A and B.
    %
    Note that the location on the WS$_2$ nanoflowers where each spectra has been recorded
    was indicated in Fig.~\ref{fig:ws2positions}.
  }
   \label{table:sampledata}
\end{table}
%%%%%%%%%%%%%%%%%%%%%%%%%%%%%%%%%%%%%%%%%%%%%%%%%%%%%%%%%%%%%%%%%%%%%%%%%%%%%%%%%%%%%%%%%%%%%%%%%5
%%%%%%%%%%%%%%%%%%%%%%%%%%%%%%%%%%%%%%%%%%%%%%%%%%%%%%%%%%%%%%%%%%%%%%%%%%%%%%%%%%%%%%%%%%%%%

In the following we will present results for representative spectra
corresponding to each of the two samples.
%
The full set of spectra together are available together with {\tt EELSfitter},
the code used to produce the results of this analysis
and whose installation
and usage instructions are presented in Appendix~\ref{sec:installation}.

\subsection{Subtraction procedure}

In Table~\ref{table:sampledata_summary} we collect
the mean value and uncertainty of the first local minimum, $\Delta E|_{\rm min}$.
averaged over the spectra corresponding to samples A and B from
Fig.~\ref{fig:ws2positions}.
%
The location of the first minimum is relatively stable
among all the spectra belonging to a given set.
%
This indicates that the onset of the inelastic contributions $I_{\rm inel}$ does
not change significantly as we move to different regions of the sample.
%
We also indicate
the corresponding values of the hyper-parameters
$\Delta E_{\rm I}$ and $\Delta E_{\rm II}$ defined in Fig.~\ref{fig:EELS_toy}.
%
Recall that only
the data points with $\Delta E \le \Delta E_{\rm I}$ is used for the training
of the neural network model.
%
For $\Delta E \ge \Delta E_{\rm II}$ instead, the training set includes only the pseudo-data
that implements the $I_{\rm ZLP}(\Delta E)\to 0$ constraint.
 %
The model training is performed for a range of $\Delta E_{\rm I}$ values
subject to the condition that $\Delta E_{\rm I} \le \Delta E_{\rm min}$.

%%%%%%%%%%%%%%%%%%%%%%%%%%%%%%%%%%%%%%%%%%%%%%%%%%%%%%%%%%%%%%%%%%%%%%%%%%%%%%%%%%%%%%%%%%%%%
%%%%%%%%%%%%%%%%%%%%%%%%%%%%%%%%%%%%%%%%%%%%%%%%%%%%%%%%%%%%%%%%%%%%%%%%%%%%%%%%%%%%%%%%%%%%%
\begin{table}[t]
  \begin{center}
            \renewcommand{\arraystretch}{1.50}
  \begin{tabular}{@{}ccccccccc}
\br
Set & $\Delta E|_{\rm min}$~(eV)  &  $\Delta E_{\rm I}$~(eV)  &  $\Delta E_{\rm II}$~(eV)   \\
\mr
A        &    $2.70\pm0.06$               &          1.8        &      12         \\
B        &    $1.80\pm0.04$               &          1.4        &      6        \\
\br
  \end{tabular}
    \end{center}
  \caption{\small The mean value and uncertainty of the first local minima, $\Delta E|_{\rm min}$,
    averaged over the spectra corresponding to samples A and B from
    Fig.~\ref{fig:ws2positions}.
    %
    We also indicate
     the corresponding values of the hyper-parameters
     $\Delta E_{\rm I}$ and $\Delta E_{\rm II}$ defined in Fig.~\ref{fig:EELS_toy} used for the training
     of the neural network model.
    %
  }
   \label{table:sampledata_summary}
\end{table}
%%%%%%%%%%%%%%%%%%%%%%%%%%%%%%%%%%%%%%%%%%%%%%%%%%%%%%%%%%%%%%%%%%%%%%%%%%%%%%%%%%%%%%%%%%%%%%%%%5
%%%%%%%%%%%%%%%%%%%%%%%%%%%%%%%%%%%%%%%%%%%%%%%%%%%%%%%%%%%%%%%%%%%%%%%%%%%%%%%%%%%%%%%%%%%%%

The optimal values of $\Delta E_{\rm I}$ listed  in Table~\ref{table:sampledata_summary} are
determined as follows.
%
We evaluate the ratio
between the derivative of the intensity distribution acquired on the specimen over the
same quantity recorded in vacuum,
\be
\label{eq:rder}
\mathcal{R}^{(j)}_{\rm der}(\Delta E) \equiv
\la
\frac{
  dI_{\rm EEL}^{({\rm exp})(j)}(\Delta E)/ d\Delta E
}{
  dI_{\rm EEL}^{({\rm exp})(j')}(\Delta E) /d\Delta E
} \ra_{N_{\rm sp}' } \, ,
\ee
where $j'$ labels one of the $N_{\rm sp}'$ vacuum spectra and the average is taken
over all available values of $j'$.
%
This ratio allows to identify a suitable value of $\Delta E_{I}$ by establishing
for which energy losses the shape (rather than the absolute value) of the intensity distributions 
recorded on the specimen starts to differ significantly from their vacuum counterparts.
%
A sensible choice of $\Delta E_{\rm I}$ could be for instance given by
$\mathcal{R}_{\rm der}(\Delta E_{\rm I}) \simeq 0.8$, for which derivatives differ
at the 20\% level.
%
Note also that the leftmost value of the energy loss satisfying
$\mathcal{R}_{\rm der}(\Delta E)=0$ in Eq.~(\ref{eq:rder}) corresponds to the position of the first
local minimum of the spectrum.

The end result of the  neural network training is
 a set of $N_{\rm rep}=500$ replicas
  parametrising the ZLP, $I_{\rm ZLP}^{({\rm mod})(k)}(\Delta E)$.
 %
 Taking into account that we have $N_{\rm sp}$ individual spectra in each sample,  the ZLP
 subtraction is performed individually
 for each Monte Carlo replica,
 \be
 \label{eq:subtractedModelPrediction}
 I_{\rm inel}^{({\rm exp})(j,k)}(\Delta E) \equiv I_{\rm EELS}^{({\rm exp})(j)}(\Delta E) - I_{\rm ZLP}^{({\rm mod})(k)}(\Delta E)\, ,
 \quad \forall~N_{\rm rep} \, ,\quad j=1,\ldots,N_{\rm sp} \, ,
 \ee
 from which statistical estimators can be evaluated as usual.
 %
 For instance, the mean value for our model prediction for the $j$-th spectrum
 can be evaluated by averaging over the set of replicas,
 \be
 \la  I_{\rm inel}^{({\rm exp})({\rm (exp)}j)}\ra (\Delta E)
 = \frac{1}{N_{\rm rep}} \sum_{k=1}^{N_{\rm rep}}  I_{\rm inel}^{({\rm mod})(j,k)}(\Delta E) \, ,
 \quad j=1,\ldots,N_{\rm sp} \, ,
 \ee
 and likewise for the corresponding uncertainties or correlations.
%
 For large values of $\Delta E$
 the model prediction reduces to the original spectra, since in that region
 the ZLP contribution vanishes,
 \be
 I_{\rm inel}^{({\rm exp})(j,k)}(\Delta E \gg \Delta E_{\rm I}) \to  I_{\rm EE}^{{\rm (exp)}(j)}(\Delta E) \, ,\quad
 \forall~j,k \, .
 \ee
 
 For very small values of the energy loss, the contribution to the total
 spectra from inelastic scatterings is negligible
 and thus the subtracted model prediction Eq.~(\ref{eq:subtractedModelPrediction}) should
 vanish.
 %
 However, this will not be the case in practice since the neural-net model is trained on
 the $N_{\rm sp}$ ensemble of spectra, rather that just on individual ones, and thus the expected
 $\Delta E \to 0$ behaviour will only be achieved within uncertainties rather than at the level of
 central values.
 %
 To achieve the desired $\Delta E \to 0$ limit, we apply a matching procedure
 as follows.
 %
 We introduce another hyper-parameter, $\Delta E_0 < \Delta E_1$, such that
 one has for the $k$-th ZLP replica associated to the $j$-th spectrum the following
 behaviour:
 \bea
 \nonumber
 I_{\rm ZLP}^{({\rm mod})(j,k)}(\Delta E) &=& I_{\rm EELS}^{({\rm exp})(j)}(\Delta E) \, ,\quad \Delta E < \Delta E_0  \, ,\\
 I_{\rm ZLP}^{({\rm mod})(j,k)}(\Delta E) &=& I_{\rm EELS}^{{\rm (exp)}(j)} + \lp \xi_1^{(n_l)(k)}(\Delta E) -
 I_{\rm EELS}^{{\rm (exp)}(j)}(\Delta E)\rp  \times \mathcal{F} \, , \nonumber \quad 
 \Delta E_0 < \Delta E \le \Delta E_1 \, ,\\
 &&\mathcal{F} = \exp\lp -\frac{\lp \Delta E - \Delta E_1 \rp^2 }{\lp \Delta E_0 - \Delta E_1 \rp^2 \delta^2} \rp  \, , \label{eq:matching} \\
 I_{\rm ZLP}^{({\rm mod})(j,k)}(\Delta E) &=& \xi_1^{(n_l)(k)}(\Delta E) \, , \quad \Delta E > \Delta E_1 \nonumber \, ,
 \eea
 where $\xi_1^{(n_l)(k)}$ indicates the output of the $k$-th neural network that parametrises
 the ZLP and $\delta$ is a dimensionless tunable parameter.
 %
 In Eq.~(\ref{eq:matching}), $\mathcal{F}(\Delta E)$ represents a matching factor
 that ensures that the ZLP model prediction smoothly interpolates
 between $\Delta E=\Delta E_0$ (where $\mathcal{F}\ll 1$ and the original spectrum should
 be reproduced) and $\Delta E=\Delta E_1$
 (where $\mathcal{F}=1$ leaving the neural network output unaffected).
 %
 Here we adopt $\Delta E_0 = \Delta E_1 -0.5\,{\rm eV}$,  having verified
 that results are fairly independent of this choice.
 %
 Taking into account the matching procedure, we can slightly modify Eq.~(\ref{eq:subtractedModelPrediction})
 to 
 \be
 \label{eq:subtractedModelPrediction2}
 I_{\rm inel}^{({\rm mod})(j,k)}(\Delta E) \equiv I_{\rm EELS}^{({\rm exp})(j)}(\Delta E) - I_{\rm ZLP}^{({\rm mod})(j,k)}(\Delta E)\, ,
 \quad \forall~N_{\rm rep} \, ,\quad j=1,\ldots,N_{\rm sp} \, .
 \ee

 The ensemble of ZLP-subtracted spectra $\{I_{\rm inel}^{({\rm mod})(j,k)} \} $
 can then be used to estimate the bandgap of the specimen in the region where
 they were acquired.
 %
 Different approaches  have been put forward to estimate the value of the bandgap from 
subtracted EEL spectra, \textit{e.g.} by means of the inflection point of the rising intensity or
a linear fit to the maximum positive slope~\cite{Schamm:2003}.
%
Here we will adopt the approach of~\cite{Rafferty:2000} where the behaviour
of $I_{\rm inel}(\Delta E)$ in the onset region is modeled as
\begin{equation}
  \label{eq:I1}
    I_{\rm inel}(\Delta E) \simeq  A \lp \Delta E-E_{\rm BG} \rp^{b} \, , \quad \Delta E \ge E_{\rm BG} \, ,
\end{equation}
and vanishes for $E < E_{\rm BG}$, where both the bandgap value
$E_{\rm BG}$ as well as the parameters $A$ and $b$ are extracted from the fit.
%
The exponent $b$ is expected to be $b\simeq 1/2~(3/2)$ for a semiconductor material characterised
by a direct~(indirect) bandgap.
 %
 For each of the $N_{\rm sp}$ spectra and the $N_{\rm rep}$ replicas
 we fit to Eq.~(\ref{eq:subtractedModelPrediction2}) the model Eq.~(\ref{eq:I1})
 within a range taken to be
 $\lc \Delta E_{\rm I} - 0.5~{\rm eV}, \Delta E_{\rm I} + 0.7~{\rm eV}\rc$.
 %
 One ends up with $N_{\rm rep}$ values for
 the bandgap energy and fit exponent for each spectra,
 \be
 \Big \{ E_{\rm BG}^{(j,k)}, b^{(j,k)} \Big\} \, , \quad k=1,\ldots,N_{\rm rep} \, ,
 \quad j=1,\ldots,N_{\rm sp} \, ,
 \ee
 from which again one can readily evaluate their statistical estimators.
 %
 In the following, we will display the median and the 68\% confidence level intervals
 for these parameters to account for the fact that their distribution will be in general non-Gaussian.

 \subsection{Bandgap analysis of sample A}

We present first the results of the bandgap analysis of sample A,
taking location sp14 in Fig.~\ref{fig:ws2positions} as representative spectrum; compatible results
are found for the rest of locations.
%
As mentioned above, this region is characterised by a sizable thickness where
WS$_2$ is expected to behave as a bulk material.
%
The left panel of Fig.~\ref{fig:sp4_subtracted_spectrum} displays the original
and subtracted EEL spectrum
together with the predictions of the ZLP model, where
the bands indicate the 68\% confidence level uncertainties and the central value
is the median of the distribution.
%
The inset shows the result of the polynomial fits using Eq.~(\ref{eq:I1}) to the subtracted spectrum
together with the corresponding uncertainty bands.

%%%%%%%%%%%%%%%%%%%%%%%%%%%%%%%%%%%%%%%%%%%%%%%%%%%%%%%%%%%%%%%%%%%%%%%
\begin{figure}[t]
\begin{centering}
  \includegraphics[width=0.99\linewidth]{plots/SubtractedEELS_plot_sp14.pdf}
   \caption{Left: the original
     and subtracted EEL spectrum corresponding to location sp14 of sample A in Fig.~\ref{fig:ws2positions},
     together with the predictions of the ZLP model, where
     the bands indicate the 68\% confidence level uncertainties.
     %
     The inset displays the result of fitting Eq.~(\ref{eq:I1}) to the onset
     region of the subtracted spectrum.
     %
     Right: the average ratio of the derivative of the intensity
     distribution in sp14 over its vacuum counterpart, Eq.~(\ref{eq:rder})
  }
\label{fig:sp4_subtracted_spectrum}
\end{centering}
\end{figure}
%%%%%%%%%%%%%%%%%%%%%%%%%%%%%%%%%%%%%%%%%%%%%%%%%%%%%%%%%%%%%%%%%%%%%%%%%%

One can observe how the ZLP model uncertainties are small at low $\Delta E$
(due to the matching condition) and large $\Delta E$ (where the ZLP vanishes),
but become significant in the intermediate region where the contributions
from $I_{\rm ZLP}$ and $I_{\rm inel}$ become comparable.
%
It is worth emphasizing that these (unavoidable) uncertainties are neglected in most
ZLP subtraction methods.
%
The validity of our choice for the hyperparameter $\Delta E_{\rm I}$ (Table~\ref{table:sampledata_summary})
can be verified {\it a posteriori} by evaluating the ratio
\be
\mathcal{R}^{(j)}_{\rm abs}\lp \Delta E_{\rm I}\rp \equiv 
\la I_{\rm ZLP}^{({\rm mod})(j)}\ra_{\rm rep} \Big/I_{\rm EEL}^{({\rm exp})(j)} \Big|_{\Delta E = \Delta E_{\rm I}} \, ,
\ee
which in this case turns out to be $\mathcal{R}_{\rm abs} = 0.98$.
%
It is indeed important to verify that $\mathcal{R}_{\rm abs}\lp \Delta E_1\rp$ is not too far from unity,
indicating that the training dataset has not been contaminated by the inelastic contributions.

The average ratio of the derivative of the intensity
distribution in sp14 over its vacuum counterpart, Eq.~(\ref{eq:rder}), is shown
in the right panel of  Fig.~\ref{fig:sp4_subtracted_spectrum}. 
%
From this comparison we can see that Fig.~\ref{fig:sp4_subtracted_spectrum} validates our choice of
$\Delta E_{\rm I}$.

Fig.~\label{fig:subtracted_spectra_comp} displays the
The ZLP-subtracted spectra corresponding to locations \#4, \#5, and \#6
    in ..... with the corresponding uncertainties.
    %
    Results are shown for two values of the hyperparameter $\Delta E_{\rm I}$,
    1.45 eV (left) and 1.55 eV (right panel).

%%%%%%%%%%%%%%%%%%%%%%%%%%%%%%%%%%%%%%%%%%%%%%%%%%%%%%%%%%%%%%%%%%%%%%%
\begin{figure}[t]
\begin{centering}
  \includegraphics[width=0.99\linewidth]{plots/subtracted_spectra_comp.pdf}
  \caption{The ZLP-subtracted spectra corresponding to locations \#4, \#5, and \#6
    in ..... with the corresponding uncertainties.
    %
    Results are shown for two values of the hyperparameter $\Delta E_{\rm I}$,
    1.45 eV (left) and 1.55 eV (right panel).
  }
\label{fig:subtracted_spectra_comp}
\end{centering}
\end{figure}
%%%%%%%%%%%%%%%%%%%%%%%%%%%%%%%%%%%%%%%%%%%%%%%%%%%%%%%%%%%%%%%%%%%%%%%%%%

\subsection{Bandgap determination}



We now move to discuss the results on the bandgap determination obtained
by fitting the functional form Eq.~(\ref{eq:I1}) to each of the subtracted
spectra defined by Eq.~(\ref{eq:subtractedModelPrediction2}).
%
Results will be presented as a function of the hyper-parameter $\Delta E_{\rm I}$
in order to gauge the stability of the results.
%
To begin with, in Fig.~\ref{fig:bvalues}
we display the values of the bandgap energy $E_{\rm BG}$ (top panels)
and of the exponent $b$ (bottom panels) as a function of $\Delta E_I$
for locations \#4 (left)
and \#5 (right panels) from Sample A indicated in Fig.~\ref{fig:ws2positions}.
%
The central value and the error band for each value of $\Delta E_I$ is evaluated
as the median and the 68\% CL interval over the $N_{\rm rep}=500$ Monte Carlo replicas.
%
The red marker indicates the position of the optimal value of
$\Delta E_{\rm I}$ determined as specified above.

%%%%%%%%%%%%%%%%%%%%%%%%%%%%%%%%%%%%%%%%%%%%%%%%%%%%%%%%%%

\begin{figure}[t]
\begin{centering}
  \includegraphics[width=0.6\linewidth]{plots/Stability_plots_sp4.pdf}
  \includegraphics[width=0.6\linewidth]{plots/Stability_plots_sp14.pdf} 
  \caption{Top: the values of the bandgap energy $E_{\rm BG}$ and of the exponent $b$
  obtained from fits to the onset
  region of subtracted spectra using Eq.~(\ref{eq:I1}) as a function
  of the hyper-parameter $\Delta E_{\rm I}$.
  %
  We show results for locations \#4 from set A (top)
  and \#14 of set B (bottom) indicated in Fig.~\ref{fig:ws2positions}.
  %
  The central value and the error band for each value of $\Delta E_I$ is evaluated
  as the median and the 68\% CL interval over the $N_{\rm rep}=500$ Monte Carlo replicas.
  }
\label{fig:bvalues}
\end{centering}
\end{figure}
%%%%%%%%%%%%%%%%%%%%%%%%%%%%%%%%%%%%%%%%%%%%%%%%%%%%%%%%%%%%%

In Table~\ref{table:bandgap_fitting} we collect
 the median values and 68\% CL ranges for the bandgap energy $E_{\rm BG}$
 and the bandgap exponent $b$ determined from fitting Eq.~(\ref{eq:I1}) to each of the subtracted
 spectra defined by Eq.~(\ref{eq:subtractedModelPrediction2}).
 %
 We consider two representative spectra from sample A and two
 from sample B. 
 %
 The error is divided into the statistical and the systematic component, which are
 then added in quadrature to evaluate the total uncertainty in the fit results. 
 %
 From these results we see that the spectra in sample A are consistent with a direct bandgap,
 while those of sample B with an indirect bandgap.

%%%%%%%%%%%%%%%%%%%%%%%%%%%%%%%%%%%%%%%%%%%%%%%%%%%%%%%%%%%%%%%%%%%%%%%%%%%%%%%%%%%%%%%%%%%%%
%%%%%%%%%%%%%%%%%%%%%%%%%%%%%%%%%%%%%%%%%%%%%%%%%%%%%%%%%%%%%%%%%%%%%%%%%%%%%%%%%%%%%%%%%%%%%
\begin{table}[t]
  \begin{center}
    \footnotesize
            \renewcommand{\arraystretch}{1.50}
  \begin{tabular}{@{}c|c|c|c}
\br
Set & Spectrum   &$E_{\rm BG}$~(eV)  &  $b$  \\
\mr
\mr
A        &   sp\#4   &     $ 2.0 \pm 0.3^{\rm (stat)} \pm  0.2^{\rm (sys)}=  2.0 \pm 0.3^{\rm (tot)}   $                &       $ 0.5 \pm 0.3^{\rm (stat)} \pm  0.2^{\rm (sys)}=  0.5 \pm 0.3^{\rm (tot)}   $                       \\
\mr
A        &   sp\#5   &     $ 2.0 \pm 0.3^{\rm (stat)} \pm  0.2^{\rm (sys)}=  2.0 \pm 0.3^{\rm (tot)}   $                &       $ 0.5 \pm 0.3^{\rm (stat)} \pm  0.2^{\rm (sys)}=  0.5 \pm 0.3^{\rm (tot)}   $                       \\
\mr
\mr
B        &   sp\#14   &     $ 2.0 \pm 0.3^{\rm (stat)} \pm  0.2^{\rm (sys)}=  2.0 \pm 0.3^{\rm (tot)}   $                &       $ 0.5 \pm 0.3^{\rm (stat)} \pm  0.2^{\rm (sys)}=  0.5 \pm 0.3^{\rm (tot)}   $                       \\
\mr
B        &   sp\#15   &     $ 2.0 \pm 0.3^{\rm (stat)} \pm  0.2^{\rm (sys)}=  2.0 \pm 0.3^{\rm (tot)}   $                &       $ 0.5 \pm 0.3^{\rm (stat)} \pm  0.2^{\rm (sys)}=  0.5 \pm 0.3^{\rm (tot)}   $                       \\
\br
  \end{tabular}
    \end{center}
  \caption{\small The median values and 68\% CL ranges for the bandgap energy $E_{\rm BG}$
    and the bandgap exponent $b$ determined from fitting Eq.~(\ref{eq:I1}) to each of the subtracted
    spectra defined by Eq.~(\ref{eq:subtractedModelPrediction2}).
    %
    As justified in the text, we consider two representative spectra from sample A and two
    from sample B. 
    %
    The error is divided into the statistical and the systematic component, which are
    then added in quadrature to evaluate the total uncertainty in the fit results. {\rm ToDo}.
  }
   \label{table:bandgap_fitting}
\end{table}
%%%%%%%%%%%%%%%%%%%%%%%%%%%%%%%%%%%%%%%%%%%%%%%%%%%%%%%%%%%%%%%%%%%%%%%%%%%%%%%%%%%%%%%%%%%%%%%%%5
%%%%%%%%%%%%%%%%%%%%%%%%%%%%%%%%%%%%%%%%%%%%%%%%%%%%%%%%%%%%%%%%%%%%%%%%%%%%%%%%%%%%%%%%%%%%%
