\section{Transition metal dichalcogenides}
\label{sec:tmd}

Over the past few years the exploration of 2D layered materials
has developed rapidly. 
%
In particular, significant attention has been 
going to monolayers of transition metal dichalcogenides (TMDs),
atomically thin semiconductors of the type MX$_2$, where M is a 
transition metal atom and X a chalcogen atom. 
%

The electronic structure of TMDs strongly depends on the coordination 
of the transition metal atoms, giving rise to an array of electronic
and magnetic properties~\cite{Chhowalla:2013}.
%
TMDs exhibit the interesting property to have an indirect
band gap in bulk form, whereas in monolayer form the gap becomes
direct~\cite{Splendiani:2010}.
%
The tunable electronic structure and the potential applications in
electronics makes these materials highly attractive for research. 
%
The indirect-to-direct bandgap transition is manifested as enhanced
photoluminescence in monolayers of, among others, tungsten disulfide
(WS$_2$), whereas only little emission is observed in bulk form.
%
WS$_2$ is an emergent class of TMDs, which adopts a layered structure 
by stacking atomic layers of S-W-S in a sandwich configuration. 
%
It is studied for potential applications such as storage of hydrogen 
and lithium~\cite{Bhandavat:2012}.
%
Although the interaction between the layers is a weak Van der Waals 
force, the dependence of the interlayer interaction on the stacking 
order of WS$_2$ is significant. Therefore, modulating the electronic
structure in a well-controlled way is crucial for application to
nanodevices.
%
A phenomenon called polytypism is an important issue for the interlayer
interactions within WS$_2$: different stacking types tend to coexist, 
complicating the characterization of the physical properties~\cite{Na:2018}.
%
One response of different stacking patterns to electric fields is
spontaneous electrical polarization, leading to modifications on the 
electronic band structure and correspondingly on the band gap~\cite{Li:2016}.











