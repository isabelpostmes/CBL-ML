\section{Introduction}
\label{sec:introduction}

Electron energy-loss spectroscopy (EELS) within the transmission electron microscope (TEM) provides unique information on the structural, chemical, and electronic properties of materials at the nanoscale.
%
Thanks to recent instrumentation breakthroughs
such as electron monochromators and aberration correctors,
modern EELS analyses can map these properties with unprecedented spatial and spectral resolution.
%
This unique combination makes possible for instance charting the local
electronic properties of nanomaterials
down to the single atom scale, and explore this way a number of
important phenomena
from bulk and surface plasmons, excitons,
and phonons to intra-band transitions.
%
A particularly relevant region of the EEL spectra is
the low-loss region, corresponding to electrons that have lost
less than a few eV following their inelastic interactions
with the sample.
%
This low-loss region contains a plethora of interesting
information and allows, among other applications,
to determine locally the bandgap of nanomaterials~\cite{Stoger:2008}.

Provided the studied sample is electron-transparent,
in EELS analyses
the bulk of the incident electron beam will traverse the sample
either without interacting or restricted to elastic scatterings.
%
In both cases, the resulting energy exchange is too small
to be measured in EEL spectra, leading to a 
 high intensity peak centered at energy losses
of $\Delta E\simeq 0$ and known as the zero loss peak (ZLP).
%
The energy resolution of EELS analyses is ultimately determined by
the electron beam size of the system, often expressed in terms
of the full width at half maximum (FWHM) of the
ZLP~\cite{Egerton:2009}.
%
In the low-loss region, the contribution from the ZLP
typically overwhelms the sample contribution arising
from inelastic scatterings.
%
For this reason, important signals of relevant low-loss phenomena such as excitons,
phonons, and intraband transitions risk being drowned
in the tails of the ZLP~\cite{Abajo:2010}.
%
Therefore, the accurate removal of the ZLP
contribution is crucial  in order to  chart the  features
of the low-loss EELS region. 

The properties of the ZLP in monochromated EELS analyses depend mainly on the electron energy dispersion, the monochromator alignment, and the sample thickness~\cite{Park:2008, Stoger:2008}.
%
The first two factors arise already in the absence of a specimen, while the third one is associated
to elastic scatterings with the sample such as  phonon excitation and exciton losses.
%
For this reason, measurements of EEL spectra can only be used for calibration purposes
but are not suitable
to subtract the ZLP from spectra taken on specimens, since their shapes will be in general
different.

Several approaches to ZLP subtraction have been put forward in the literature.
%
These are based on specific model assumptions about the ZLP properties, specifically
concerning its parametric functional dependence, from Lorentzian~\cite{Dorneich:1998}
and power laws~\cite{Erni:2005} to more general multiple-parameter functions~\cite{Benthem:2001}.
%
Another approach is based on the mirroring the $\Delta E <0$ region of the spectra, assuming
that the $\Delta E>0$ region is fully symmetric~\cite{Lazar:2003}.
%
These  subtraction methods are however affected by three main limitations.
%
Firstly, they rely on specific model assumptions {\it e.g.} with
the choice of functional form, introducing a methodological
bias whose size is difficult to quantify.
%
Secondly, they lack an estimate of the associated uncertainties, which in turn affects
the reliability of any physics interpretations of the low loss region such as the band gap extraction.
%
Thirdly, manual choices of such as those of the fitting ranges introduce a significant degree of
arbitrariness in the procedure.

In this work we bypass these limitations by developing a model-independent strategy
for ZLP subtractions in EELS analysis by means of machine learning.
%
Our strategy is based on the so-called NNPDF approach~\cite{Ball:2008b,Ball:2012cx,Ball:2014uw,Ball:2017},
developed for studies
of the quark and gluon structure of the proton in high-energy particle collisions.
%
The main idea is to combine the  Monte Carlo replica  method to construct a probability
distribution in the space of experimental data with artificial
neural networks as unbiased interpolator to describe the ZLP and the associated
uncertainties.
%
The end result is a faithful sampling of the probability distribution in the space of ZLP,
which then can be applied to subtract its contribution to EEL spectra while keeping
full track of all the data, model, and parametrisation uncertainties.
%
Thanks to its model independence we can assemble high-dimensionality ML models
of the ZLP spectra with multiple inputs, and extrapolate reliably the model
to other operation conditions of the microscope beyond those included
in the training dataset.
%
While machine learning techniques have been deployed for several EM-related
applications, from to , this is the first time that they are used as unbiased
background removal applications.



This work is separated into two chapters: in the first, we will reconstruct the vacuum zero loss peak through discrete sampling, without making assumptions on its functional form. This can be best addressed using neural networks as unbiased interpolants. This way we develop a generalized N-dimensional model to predict the shape of the zero loss distribution, based on an arbitrary number N input variables. 
In the second chapter, we switch from vacuum to on-sample recorded EEL spectra and we use a three-input neural network to fit and predict the intensity of the ZLP. By subtraction, we can find the relevant underlying spectra and look for the bandgap energy.
The general strategy in both chapters will involve two stages. In the first stage, one generates a set Monte Carlo pseudo data on the original ZLP data - either recorded in vacuum, or extracted from the in-sample spectrum. This ensemble is generated such that it is large enough to reproduce the underlying statistical properties of the original dataset. In the second stage, on each Monte Carlo replica an individual neural network will be trained. The experimental predictions will be different in each replica, and the ensemble of the best predictions will be used for the computation of physical observables.\\
It is especially important to define the way to which the best prediction is determined: the best fit should be independent of the network parametrization, and also establishing the best fit needs to be done with care. Both aspects will be justified explicitly in the methodology section. 


This paper is organised as follows.

The paper is organized as follows. For each of the chapters, in sect. 2 we summarize the features of the experimental data. In sect. 3 we verify that the Monte Carlo sample represents the data faithfully, we summarize the methods we use to prepare and fit the neural network, we describe the minimization and validation methods and we discuss the interpolation and extrapolation. In the second chapter, we include the methods that need to be taken for the subtraction and band gap determination. Supporting figures and tables will be presented in the Appendix.
In sect. 4 we present in detail the results: we discuss the success of the interpolation and extrapolation in Chapter 1 and the success of the subtraction and bandgap determination in Chapter 2. Finally we provide a general summary and outlook on future developments in section 5. 

