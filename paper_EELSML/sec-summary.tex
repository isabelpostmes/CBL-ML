%%%%%%%%%%%%%%%%%%%%%%%%%%%%%%%%%%%%%%%%%
\section{Summary and outlook}
%%%%%%%%%%%%%%%%%%%%%%%%%%%%%%%%%%%%%%%%%
\label{sec:summary}

In this work we have presented a novel, model-independent strategy to parametrise and subtract
the ubiquitous zero-loss peak that dominates  the low-loss region
of EEL spectra.
%
Our strategy is based on machine learning and provides a faithful estimate of the
uncertainties associated to both the input data and the procedure itself,
which can  then be propagated to physical predictions  without any  approximations.
%
We have demonstrated how, in the case of vacuum spectra, our approach
is sufficiently flexible to accomodate several input variables corresponding
to different operation conditions of the microscope.
%
Further, we are able  to reliably
extrapolate our predictions, {\it e.g.} for the  expected FWHM of the ZLP,
to other operation conditions.
%
When applied to spectra recorded on specimens, our approach
makes possible to robustly disentangle the ZLP contribution from
those arising from inelastic scatterings.
%
Thanks to this subtraction, one can fully exploit
the valuable physical information contained in the ultra low-loss region of
the spectra.

We have applied this ZLP subtraction
strategy to EEL spectra recorded in  WS$_2$ nanoflowers characterised by a
2H/3R polytypic crystalline structure.
%
First of all, measurements taken in a relatively
thick region of the specimen were used to estimate the local value of the bandgap energy $E_{\rm BG}$
and determine whether it is direct or indirect.
%
A model fit to the onset of the inelastic distribution finds $E_{\rm BG} \simeq 1.6 \pm 0.2\,{\rm eV}$ and
a clear preference for an indirect bandgap, irrespective of the specific location
of the specimen where the  spectra were recorded.
%
These findings are consistent with previous studies, both of theoretical
and of experimental nature, of the bandgap structure of bulk WS$_2$.

Further, we have also applied to our method to a much thinner region of the  WS$_2$ nanoflowers,
specifically one
composed of overlapping petals whose thicknesses can be as small as a few monolayers.
%
Instead of studying the bandgap properties, one could exploit the ZLP-subtracted results 
of this sample to study the local
excitonic transitions that are observed in the ultra-low-loss region of the spectra.
%
The ZLP-subtracted spectra in this sample have allowed
the charting of exciton peaks down to $\Delta E\simeq 1.5$ eV together with
the associated uncertainty estimate, allowing to establish their significance.


The approach presented in this work could be extended
in several directions.
%
First of all, it would be interesting to test its robustness when additional
operation conditions of the microscope are included as input variables,
and to assess to which extend ZLP models obtained with an specific microscope
can be generalised to an altogether different TEM.
%
Further, a strong cross-check of our method would be provided by comparing
our predictions for other operation conditions of the microscope, such
as the FWHM as a function of the beam energy $E_b$ reported in Fig.~\ref{fig:extrapolbeam}
with actual measurements and verifying whether or not there is agreement within the
uncertainties of the prediction.
%
Concerning the physical interpretation of the low-loss region of EEL
spectra, our method could be applied to study the bandgap
and other local electronic properties of different types
of nanostructures built upon 2D layered materials, such as MoS$_2$ nanowalls
and nano-pillars and WS$_2$/MoS$_2$ arrays or heterostructures.
%
In addition to the bandgap characterisation, one might
consider the implications of our approach for the study
of other phenomena relevant for the interpretation of the low-loss
region such as  plasmons, excitons, phonon interactions, and
intra-band transitions.
%
One could   further exploit the subtracted EEL spectra produced
with our method to evaluate the complex dielectric function and its associated
uncertainties by means of the Kramers-Kronig relation.
%
Such phenomenological studies of the local electronic properties would be compared
with {\it ab initio} calculations such as Density Functional Theory, based
on the same underlying crystalline structure of the analysed samples.
%
We recall that the results
presented in this work are to the best of our knowledge the first EELS bandgap
analysis of WS$_2$ nanostructures based on mixed 2H/3R polytypes.

Another possible generalisation of our method would be to the study of spectral TEM images,
where each pixel in the image contains an individual EEL spectrum (possibly
extended to 3D images).
%
Here machine learning methods would be useful in order
to  identify relevant features of the spectra (peaks, edges, shoulders) in a fully
automated way
without having to process each spectrum individually, and then assess
how these features vary as we move along different regions of the
nanostructure.
%
Such application would combine two of the most topical uses of machine learning, regression
on the one hand and classification on the other hand.

\subsection*{Acknowledgments}
%
The work of J.~R. has been partially supported by the
Dutch Research Council (NWO).
%
We are grateful to Emanuele R. Nocera and Jake Ethier for
assistance in installing the code in the Nikhef cluster.
