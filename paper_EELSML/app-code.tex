
\section{Installation and usage of {\tt EELSfitter}}
\label{sec:installation}

In this appendix we provide some instructions about the installation
and the usage of the {\tt EELSfitter} code developed
in this work.
%
The code is publicly available from its GitHub repository
\begin{center}
\url{https://github.com/LHCfitNikhef/EELSfitter}.
\end{center}

\subsection*{Load\_data.py}
The purpose of the first script, {\tt Load\_data.py}, is to read the spectrum
intensities and create dataframes to be used for training the neural network.
%

It reads out the spectrum intensities, automatically selects the energy loss
at which the peak intensity occurs and translates the dataset such that
the peak intensity is centered at $\Delta E =$0. 
%
Further, for each spectrum it returns the normalized intensity by normalizing
over the total area under the spectrum. 
%
The output is two datasets, {\tt df} and {\tt df\_vacuum} which contain the 
information on the in-sample and in-vacuum recorded spectra respectively. 
%
The user needs to upload the spectral data in .txt format to the 'Data' folder
and make sure that the vacuum and in-sample spectra are added to the right one
of the two datasets. 
%
For each of the spectra, the minimum and maximum value of the recorded energy 
loss need to be set manually in {\tt Eloss\_min} and {\tt Eloss\_max}.

\subsection*{Fitter.py}
This file will be used to run the neural network training on the data that was 
uploaded using the {\tt Load\_data.py} file.
%
It involves a few automatic steps to determine the hyper-parameters $\Delta E_I$
and $\Delta E_{II}$, automatically prepares and cuts the data before it is feeded
to the neural network to start the training. 
%
In order to determine the value for $\Delta E_I$, a dataframe {\tt df\_dx} is created
and it calculates the derivatives of each of the in-sample recorded spectra, 
stored as {\tt df\_dx['derivative y*']}, where {\tt *} is any of the in-sample recorded spectra.
%
Next, the first crossing of any of the derivatives with zero is determined 
and this is the value of $\Delta E_I$. 
