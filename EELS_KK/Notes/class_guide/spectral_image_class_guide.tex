\documentclass{article}
\usepackage[utf8]{inputenc}

\title{Spectral\_image class guide}
\author{isabelpostmes }
\date{January 2021}
\usepackage{verbatim}

\usepackage{hyperref}
\hypersetup{
    colorlinks=true,
    linkcolor=blue,
    filecolor=magenta,      
    urlcolor=cyan,
}

\urlstyle{same}



\usepackage{listings}
\usepackage{stmaryrd}
\usepackage{pythonhighlight}
\usepackage{comment}
\usepackage[utf8]{inputenc}
\usepackage{xcolor}
\usepackage[labelfont=bf]{caption}
\definecolor{eminence}{RGB}{128,8,130}

% Default fixed font does not support bold face
\DeclareFixedFont{\ttb}{T1}{txtt}{bx}{n}{9} % for bold
\DeclareFixedFont{\ttm}{T1}{txtt}{m}{n}{9}  % for normal

% Custom colors
\usepackage{color}
\definecolor{deepblue}{rgb}{0,0,0.5}
\definecolor{deepred}{rgb}{0.6,0,0}
\definecolor{deepgreen}{rgb}{0,0.5,0}

\usepackage{listings}

% Python style for highlighting
\newcommand\pythonstyle{\lstset{
language=Python,
basicstyle=\ttm,
otherkeywords={self},             % Add keywords here
keywordstyle=\ttb\color{deepblue},
emph={MyClass,__init__},          % Custom highlighting
emphstyle=\ttb\color{deepred},    % Custom highlighting style
stringstyle=\color{deepgreen},
frame=tb,                         % Any extra options here
showstringspaces=false            % 
}}


% Python environment
\lstnewenvironment{python}[1][]
{
\pythonstyle
\lstset{#1}
}
{}

% Python for external files
\newcommand\pythonexternal[2][]{{
\pythonstyle
\lstinputlisting[#1]{#2}}}

% Python for inline
\newcommand\pythoninline[1]{{\pythonstyle\lstinline!#1!}}





\begin{document}

\maketitle

\section{Guide}



\lstset{
language=Python,
basicstyle=\ttm,
otherkeywords={self},             % Add keywords here
keywordstyle=\ttm\color{blue},
keywordstyle=[2]\ttm\color{eminence},
keywordstyle=[3]\ttm\color{brown},
keywordstyle=[4]\ttm\color{red},
emph={MyClass,__init__},          % Custom highlighting
emphstyle=\ttb\color{deepred},    % Custom highlighting style
stringstyle=\color{deepgreen},
commentstyle=\ttb\color{gray},
numberstyle=\ttb\color{gray},
%identifierstyle=\ttm\color{purple},
morekeywords=[2]{True, False, range, bool, self},
morekeywords={as},
morekeywords=[4]{self, cls}
frame=tb,                         % Any extra options here
showstringspaces=false,
breaklines=true,
%postbreak=\mbox{\textcolor{red}{$\hookrightarrow$}\space},
}

%\lstinputlisting{image_class.py}

\lstset{emph={%  
    self, cls%
    },emphstyle={\color{red}}%
}%




Lets talk through the Spectral\_image class. We start by loading a spectral image, saved in a .dm3 or .dm4 file through:

\begin{lstlisting}
>>> im = Spectral_image.load_data('path/to/dmfile.dm4')
\end{lstlisting}

This calls on an alternative constructor, in which the data from the dm-file is loaded, and plugged into the regular constructor. In this function, the loading package \verb|ncempy.io.dm| is used,  more info \href{https://pypi.org/project/ncempy/}{here}. 



\begin{lstlisting}[numbers=left, firstnumber=81]
    @classmethod
    def load_data(cls, path_to_dmfile):
        """
        INPUT: 
            path_to_dmfile: str, path to spectral image file (.dm3 or .dm4 extension)
        OUTPUT:
            image -- Spectral_image, object of Spectral_image class containing the data of the dm-file
        """
        dmfile = dm.fileDM(path_to_dmfile).getDataset(0)
        data = np.swapaxes(np.swapaxes(dmfile['data'], 0,1), 1,2)
        ddeltaE = dmfile['pixelSize'][0]
        pixelsize = np.array(dmfile['pixelSize'][1:])
        energyUnit = dmfile['pixelUnit'][0]
        ddeltaE *= cls.get_prefix(energyUnit, 'eV')
        pixelUnit = dmfile['pixelUnit'][1]
        pixelsize *= cls.get_prefix(pixelUnit, 'm')
        image = cls(data, ddeltaE, pixelsize = pixelsize)
        return image
\end{lstlisting}

Furthermore, we see the \verb|cls.get_prefix()|, which is a small function which recognises the prefix in a unit and transfers it to a numerical value (e.g. 1E3 for k), see lines 870-916 in the complete code. Furthermore, the general constructor is called upon with

\verb|cls(data, ddeltaE, pixelsize = pixelsize)|.

The spectral image class starts by defining some constant variables, both class related and physical. The class constructor takes in at least the data of the spectral image, \verb|data|, and the broadness of the energy loss bins, \verb|deltadeltaE|. Other metadata can be given if known. 

\begin{lstlisting}[numbers=left, firstnumber=41]
class Spectral_image():
\end{lstlisting}
\begin{lstlisting}[numbers=left, firstnumber=53]
    def __init__(self, data, deltadeltaE, pixelsize = None, beam_energy = None, collection_angle = None, name = None):
        self.data = data
        self.ddeltaE = deltadeltaE
        self.determine_deltaE()
        if pixelsize is not None:
            self.pixelsize = pixelsize
        self.calc_axes()
        if beam_energy is not None:
            self.beam_energy = beam_energy
        if collection_angle is not None:
            self.collection_angle = collection_angle
        if name is not None:
            self.name = name

\end{lstlisting}


\begin{lstlisting}[numbers=left, firstnumber=100]
\end{lstlisting}

\section{Complete code}
%\lstinputlisting{../../pyfiles/image_class.py}

\end{document}
